\begin{abstract}
Recently, deep generative models have emerged as a promising method of performing \textit{de novo} drug design. They are used to discover novel compounds with desired properties. \\

In this respect, this study proposes a molecular-graph $\beta$-conditional variational autoencoders ($\beta$-CVAE) model for \textit{de novo} drug design. We perform single-objective and multi-objective optimization on two molecular properties – crippen's octanol-water partition coefficient (ClogP) and crippen's molar refractivity (CMR). We directly control these properties for optimization by assigning them as conditional vectors. Disentanglement was introduced to the latent space in our model using an arbitrary $\beta = 2$. We present and compare our results with other VAE-based models. Single-objective optimization results show that the $\beta$-CVAE generated ClogP $= 33.351\%$ and CMR $=74.871\%$ of molecules that satisfy the Ghose filter as suitable drug candidates when C1 $= 2$ and C2 $= 70$. Multi-objective optimization results show that our model generated 26.680\% for both target properties. Results suggest that the $\beta$-conditional VAE is a promising model to be applied in deep generative \textit{de novo} drug design. 
\end{abstract}