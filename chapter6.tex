\chapter{Conclusion}
In conclusion, a molecular graph-based $\beta$-CVAE deep generative model was proposed for \textit{de novo} drug design. We generated the molecular graphs through an atom-based procedure using the initial graph matrix. We conducted single-objective and multi-objective optimization using ClogP and CMR as our desired molecular properties. To do so, we directly controlled ClogP and CMR by assigning them as condition vectors to our model. Single-objective optimization results show that the $\beta$-CVAE generated ClogP $= 33.351\%$ and CMR $=74.871\%$ of molecules that satisfy the Ghose filter as suitable drug candidates when C1 $= 2$ and C2 $= 70$. Multi-objective optimization results show that our model generated 26.680\% for both targeted properties. To evaluate the $\beta$-CVAE, the results were compared with VAE, $\beta$-VAE, and CVAE models. Therefore, our deep generative model presented in this study has the potential to be applied in \textit{de novo} drug discovery after improvements in the second part of this project. Future work in this study could also optimize more than two properties. 