\chapter{Introduction}
The aim of drug discovery is to find new chemical entities with desirable pharmacological properties. This is an inherently difficult task due to the vast size and complexity of the chemical search space. The range of drug-like molecules has been estimated to be between 10\textsuperscript{23} and 10\textsuperscript{60} \cite{polishchuk2013estimation}. Meanwhile, the chemical space is discrete, making the search difficult to perform \cite{kirkpatrick2004chemical}. As a result, drug development is a lengthy and costly process, with the average clinical development time for one drug reaching more than nine years and the estimated median development cost 1.1 billion USD \cite{wouters2020estimated}.

\section{\textit{De novo} drug design}
\textit{De novo} drug design, also known as generative chemistry and \textit{de novo} molecular design, aims to reduce the time and cost of the drug development process using computer-based methods \cite{meyers2021novo}. Using computational approaches, the chemical space could be traversed more effectively to propose novel chemical structures that optimally satisfy a desired molecular profile. However, \textit{de novo} drug design remains a challenging task due to the need for human expertise and domain knowledge.

\section{Deep generative models}
Thanks to the recent developments in deep learning (DL), a surge of novel approaches has shed new light on \textit{de novo} drug design. In brief, DL refers to the use of artificial neural networks with multiple hidden layers \cite{lecun2015deep}. Gradient-based DL approaches are often pretrained on large corpora of existing molecular structures and learn to navigate arbitrary property surfaces toward optimal solutions \cite{meyers2021novo}. Within DL, a panoply of deep generative models such as Generative Adversarial Networks (GANs) and Autoencoders (AEs) have been applied to \textit{de novo} molecular design \cite{kadurin2017drugan}. These models were trained on existing data sets to capture the underlying data generation processes by sampling from an unknown probabilistic distribution \cite{sousa2021generative}. To that end, these models were capable of generating novel compounds that closely resembled those from the training data and not simply producing copies \cite{foster2019generative}. 

Deep generative models today could not only generate new molecules but also alter molecular structures to optimize specific chemical properties which are pivotal for drug discovery \cite{bilodeau2022generative}. 

\section{Paper format}
In this work, we aim to generate drug-like molecules with desired properties using the \textit{de novo} drug design approach. Firstly, this paper presents a literature review of \textit{de novo} drug design and AE models: variational autoencoders (VAE), $\beta$-VAEs, and conditional VAE (CVAE). We also present the limitations of previously published text-based AEs in \textit{de novo} drug discovery. Secondly, we introduce a molecular-graph $\beta$-CVAE model with multi-objective optimization of molecular properties - Crippen's octanol-water partition coefficient (logP) and Crippen's molar refractivity (MR). Thirdly, we compare our proposed model with the aforementioned models in our literature review and present the results. Finally, conclusions are drawn and we discuss the limitations and possible extensions to this project.